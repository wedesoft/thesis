\subsection{error.hh}\label{cha:error-hh}
\lstset{language=C++,frame=single,basicstyle=\ttfamily\bfseries\color{codegray}\scriptsize}
\begin{lstlisting}
#ifndef ERROR_HH
#define ERROR_HH
#include <exception>
#include <sstream>
class Error: public std::exception
{
public:
  Error(void) {}
  Error( Error &e ): std::exception( e )
    { m_message << e.m_message.str(); }
  virtual ~Error(void) throw() {}
  template< typename T >
  std::ostream &operator<<( const T &t )
    { m_message << t; return m_message; }
  std::ostream &operator<<( std::ostream& (*__pf)(std::ostream&) )
    { (*__pf)( m_message ); return m_message; }
  virtual const char* what(void) const throw() {
    temp = m_message.str();
    return temp.c_str();
  }
protected:
  std::ostringstream m_message;
  mutable std::string temp;
};
#define ERRORMACRO( condition, class, params, message ) \
  if ( !( condition ) ) {                               \
    class _e params;                                    \
    _e << message;                                      \
    throw _e;                                           \
  };
#endif
\end{lstlisting}
\subsection{init.cc}\label{cha:init-cc}
\lstset{language=C++,frame=single,basicstyle=\ttfamily\bfseries\color{codegray}\scriptsize}
\begin{lstlisting}
#include "malloc.hh"
#ifdef WIN32
#define DLLEXPORT __declspec(dllexport)
#define DLLLOCAL
#else
#define DLLEXPORT __attribute__ ((visibility("default")))
#define DLLLOCAL __attribute__ ((visibility("hidden")))
#endif
extern "C" DLLEXPORT void Init_malloc(void);
extern "C" {
  void Init_malloc(void)
  {
    VALUE rbHornetseye = rb_define_module( "Hornetseye" );
    Malloc::init( rbHornetseye );
    rb_require( "malloc_ext.rb" );
  }
}
\end{lstlisting}
\subsection{malloc.cc}\label{cha:malloc-cc}
\lstset{language=C++,frame=single,basicstyle=\ttfamily\bfseries\color{codegray}\scriptsize}
\begin{lstlisting}
#include "error.hh"
#include "malloc.hh"
#include <iostream>
using namespace std;
#ifdef WIN32
#define DLLEXPORT __declspec(dllexport)
#define DLLLOCAL
#else
#define DLLEXPORT __attribute__ ((visibility("default")))
#define DLLLOCAL __attribute__ ((visibility("hidden")))
#endif
VALUE Malloc::cRubyClass = Qnil;
VALUE Malloc::init( VALUE rbModule )
{
  cRubyClass = rb_define_class_under( rbModule, "Malloc", rb_cObject );
  rb_define_singleton_method( cRubyClass, "orig_new",
                              RUBY_METHOD_FUNC( mallocNew ), 1 );
  rb_define_method( cRubyClass, "orig_plus",
                    RUBY_METHOD_FUNC( mallocPlus ), 1 );
  rb_define_method( cRubyClass, "orig_read",
                    RUBY_METHOD_FUNC( mallocRead ), 1 );
  rb_define_method( cRubyClass, "orig_write",
                    RUBY_METHOD_FUNC( mallocWrite ), -1 );
  return cRubyClass;
}
VALUE Malloc::mallocNew( VALUE rbClass, VALUE rbSize )
{
  VALUE retVal = Qnil;
  try {
    unsigned int size = NUM2UINT( rbSize );
    char *self = ALLOC_N( char, size );
    ERRORMACRO( self != NULL, Error, , "Failed to allocate " << size
                << " bytes of memory" );
    retVal = Data_Wrap_Struct( rbClass, 0, xfree, (void *)self );
  } catch( std::exception &e ) {
    rb_raise( rb_eRuntimeError, "%s", e.what() );
  };
  return retVal;
}
VALUE Malloc::mallocPlus( VALUE rbSelf, VALUE rbOffset )
{
  VALUE retVal = Qnil;
  try {
    char *self; Data_Get_Struct( rbSelf, char, self );
    int offset = NUM2INT( rbOffset );
    ERRORMACRO( offset >= 0, Error, , "Offset must be non-negative (but was "
                << offset << ')' );
    retVal = Data_Wrap_Struct( cRubyClass, 0, 0, (void *)( self + offset ) );
  } catch( std::exception &e ) {
    rb_raise( rb_eRuntimeError, "%s", e.what() );
  };
  return retVal;
}
VALUE Malloc::mallocRead( VALUE rbSelf, VALUE rbLength )
{
  char *self; Data_Get_Struct( rbSelf, char, self );
  unsigned int length = NUM2UINT( rbLength );
  return rb_str_new( self, length );
}
VALUE Malloc::mallocWrite( int argc, VALUE *rbArgv, VALUE rbSelf )
{
  VALUE retVal = Qnil;
  try {
    char *self; Data_Get_Struct( rbSelf, char, self );
    ERRORMACRO( argc == 1 || argc == 2, Error, , "Malloc#write accepts "
                  "one or two arguments (not " << argc << ")" );
    if ( argc == 1 ) {
      VALUE rbString = rbArgv[0];
      memcpy( self, StringValuePtr( rbString ), RSTRING_LEN( rbString ) );
      retVal = rbString;
    } else {
      VALUE rbOther = rbArgv[0];
      VALUE rbSize = rbArgv[1];
      char *other; Data_Get_Struct( rbOther, char, other );
      int size = NUM2INT( rbSize );
      memcpy( self, other, size );
      retVal = rbOther;
    };
  } catch ( std::exception &e ) {
    rb_raise( rb_eRuntimeError, "%s", e.what() );
  };
  return retVal;
}
\end{lstlisting}
\subsection{malloc.hh}\label{cha:malloc-hh}
\lstset{language=C++,frame=single,basicstyle=\ttfamily\bfseries\color{codegray}\scriptsize}
\begin{lstlisting}
#ifndef MALLOC_HH
#define MALLOC_HH
#include "rubyinc.hh"
class Malloc
{
public:
  static VALUE cRubyClass;
  static VALUE init( VALUE rbModule );
  static VALUE mallocNew( VALUE rbClass, VALUE rbSize );
  static VALUE mallocPlus( VALUE rbSelf, VALUE rbOffset );
  static VALUE mallocRead( VALUE rbSelf, VALUE rbLength );
  static VALUE mallocWrite( int argc, VALUE *rbArgv, VALUE rbSelf );
};
#endif
\end{lstlisting}
\subsection{rubyinc.hh}\label{cha:rubyinc-hh}
\lstset{language=C++,frame=single,basicstyle=\ttfamily\bfseries\color{codegray}\scriptsize}
\begin{lstlisting}
#ifndef HORNETSEYE_RUBYINC_HH
#define HORNETSEYE_RUBYINC_HH
#ifdef RSHIFT
#undef RSHIFT
#endif
#define gettimeofday rubygettimeofday
#define timezone rubygettimezone
#include <ruby.h>
#undef timezone
#undef gettimeofday
#ifdef read
#undef read
#endif
#ifdef write
#undef write
#endif
#ifdef RGB
#undef RGB
#endif
#ifndef RUBY_VERSION_NUMBER
#define RUBY_VERSION_NUMBER ( RUBY_VERSION_MAJOR * 10000 + \
                              RUBY_VERSION_MINOR * 100 + \
                              RUBY_VERSION_TEENY )
#endif
#ifndef RUBY_METHOD_FUNC
#define RUBY_METHOD_FUNC(func) ((VALUE (*)(ANYARGS))func)
#endif
#ifndef xfree
#define xfree free
#endif
#endif
\end{lstlisting}
